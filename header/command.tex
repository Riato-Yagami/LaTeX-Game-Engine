\newcommand{\slide}[2]{
    \begin{frame}
    \frametitle[#1]{#1}
        #2
    \end{frame}
}

\newcommand{\cslide}[1]{
    \slide{}{\begin{center}#1\end{center}}
}

\newcommand{\calc}[1]{\numexpr#1\relax}
% \newcommand{\calc}[1]{\pgfmathparse{#1}\pgfmathresult}
\newcommand{\ncalc}[1]{\number\calc{#1}}
\newcommand{\drawGrid}[3]{
    \foreach \n in {0,...,#1}
        \draw[line width = #3] (\n,0) -- (\n,#2);
    \foreach \n in {0,...,#2}
        \draw[line width = #3] (0,\n) -- (#1,\n);
}

\newcommand{\drawSquare}[4]{ % 1 Offset - 2 Size - 3 Color - 4 Width
    \draw[line width = #4, color = #3, shift={#1}] (0,0) -- (0,#2) -- (#2,#2) -- (#2,0) -- cycle;
}

% \newcommand{\gameSlide}[2]{
%     \def\x{#1}\def\y{#2}
%     \level{\x}{\y}
%     \slideLink{\x}{\y}
%     \controller{\x}{\y}
% }

\newcounter{elemPos}
\newcounter{element}
\newcommand{\listElement}[2]{%
    \setcounter{elemPos}{0}% Start counting from 0
    \def\resultVal{0}% Default value
    \renewcommand*{\do}[1]{%
        \ifnumequal{\value{elemPos}}{#2}{%
            \edef\resultVal{##1}%
            \listbreak% Break out of the loop
        }{}%
        \stepcounter{elemPos}%
    }%
    % \docsvlist{#1}
    \expandafter\docsvlist\expandafter{#1}% Expand the list before passing it to \docsvlist
    % \setcounter{res}{\resultVal}%
    \setcounter{element}{\resultVal}
    % \resultVal%
    % \theelemPos
}

\newcommand{\slideLink}[2]{
    \def\x{#1}\def\y{#2}
    % \def\coordonates{\x-\y}
    \def\coordonates{\tryMove{0}{0}}
    % \coordonates
    \hypertarget{\coordonates}{}
}

\newcommand{\level}{
    % \def\x{#1}\def\y{#2}
    \begin{tikzpicture}[scale=\tikzScale]
        \drawGrid{\gridX}{\gridY}{\gridWidth};
        \drawSquare{(\x,\y)}{1}{\characterColor}{\characterWidth};
        \drawWalls;
    \end{tikzpicture}
}

\newcommand{\drawWalls}{
    \foreach \wall in \walls{
        \drawSquare{(\wall)}{1}{\wallColor}{\wallWidth};
    }
}

\NewDocumentCommand{\dividePage}{mm O{0.5}}{
    \pgfmathparse{1-#3}
    \begin{minipage}{#3\linewidth}
        #1
    \end{minipage}
    \begin{minipage}{\pgfmathresult\linewidth}
        #2
    \end{minipage}
}

\newcommand{\button}[2]{%
    \def\link{#1}%
    \hyperlink{\link}{
        \begin{tikzpicture}
            \drawSquare{(0,0)}{1}{Blue}{\buttonWidth};
            \node at (0.5,0.5) {#2};
        \end{tikzpicture}
    }
}

\newcommand{\maxValue}[2]{%
    \ifnum #1 < #2
        #2%
    \else
        #1%
    \fi
}

\newcommand{\minValue}[2]{%
    \ifnum #1 < #2
        #1%
    \else
        #2%
    \fi
}

\newcommand{\boundValue}[3]{%
    \minValue{\maxValue{#1}{#2}}{#3}%
}

\newcommand{\loopValue}[3]{%
    \ifnum #1 < #2
        #3%
    \else
        \ifnum #1 > #3
            #2%
        \else
            #1%
        \fi
    \fi
}

\newcounter{noWall}
\newcommand{\tryWall}[2]{%
    \setcounter{noWall}{1}%
    \foreach \wall in \walls{%
        \listElement{\wall}{0}%
        % \theelement
        \ifnum \theelement = \newX{#1}
            \listElement{\wall}{1}%
            \ifnum \theelement = \newY{#2}
                \setcounter{noWall}{0}%
            \fi
        \fi
    }%
    % \thetrue%
}

\newcommand{\tryMove}[2]{%
    % \tryWall{#1}{#2}%
    \ifnum \thenoWall = 1
        \move{#1}{#2}%
    \else
        \move{0}{0}%
    \fi
}

\newcommand{\newX}[1]{
    \ifnum \theloop = 1
        \loopValue{\ncalc{\x+#1}}{0}{\ncalc{\gridX-1}}%
    \else
        \boundValue{\ncalc{\x+#1}}{0}{\ncalc{\gridX-1}}%
    \fi
}

\newcommand{\newY}[1]{
    \ifnum \theloop = 1
        \loopValue{\ncalc{\y+#1}}{0}{\ncalc{\gridY-1}}%
    \else
        \boundValue{\ncalc{\y+#1}}{0}{\ncalc{\gridY-1}}%
    \fi
}

\newcommand{\move}[2]{%
    \newX{#1};\newY{#2}%
}

\newcommand{\moveButton}[3]{%
    \tryWall{#1}{#2}%
    \button{\tryMove{#1}{#2}}{$#3$}%
}

\newcommand{\controller}{
    \setlength{\tabcolsep}{-5.5pt} % Default value: 6pt
    \renewcommand{\arraystretch}{0} % Default value: 1
    \begin{tabular}{c c c}
        & \moveButton{0}{1}{\uparrow} \\
        \moveButton{-1}{0}{\leftarrow}
        & \moveButton{0}{-1}{\downarrow}
        & \moveButton{1}{0}{\rightarrow}
    \end{tabular}
}