\newcommand{\slide}[2]{
    \begin{frame}
    \frametitle[#1]{#1}
        #2
    \end{frame}
}

\newcommand{\cslide}[1]{
    \slide{}{\begin{center}#1\end{center}}
}

\newcommand{\calc}[1]{\numexpr#1\relax}
% \newcommand{\calc}[1]{\pgfmathparse{#1}\pgfmathresult}
\newcommand{\ncalc}[1]{\number\calc{#1}}
\newcommand{\drawGrid}[3]{
    \foreach \n in {0,...,#1}
        \draw[line width = #3] (\n,0) -- (\n,#2);
    \foreach \n in {0,...,#2}
        \draw[line width = #3] (0,\n) -- (#1,\n);
}

\newcommand{\drawSquare}[4]{ % 1 Offset - 2 Size - 3 Color - 4 Width
    \draw[line width = #4, color = #3, shift={#1}] (0,0) -- (0,#2) -- (#2,#2) -- (#2,0) -- cycle;
}

% \newcommand{\gameSlide}[2]{
%     \def\x{#1}\def\y{#2}
%     \level{\x}{\y}
%     \slideLink{\x}{\y}
%     \controller{\x}{\y}
% }

\newcommand{\slideLink}[2]{
    \def\x{#1}\def\y{#2}
    % \def\coordonates{\x-\y}
    \def\coordonates{\move{0}{0}}
    \coordonates
    \hypertarget{\coordonates}{}
}


\newcommand{\level}{
    % \def\x{#1}\def\y{#2}
    \begin{tikzpicture}[scale=\tikzScale]
        \drawGrid{\gridX}{\gridY}{\gridWidth};
        \drawSquare{(\x,\y)}{1}{Red}{\characterWidth};
        \drawSquare{(2,2)}{1}{Green}{\wallWidth};
    \end{tikzpicture}
}

\NewDocumentCommand{\dividePage}{mm O{0.5}}{
    \pgfmathparse{1-#3}
    \begin{minipage}{#3\linewidth}
        #1
    \end{minipage}
    \begin{minipage}{\pgfmathresult\linewidth}
        #2
    \end{minipage}
}

\newcommand{\button}[2]{
    \def\link{#1}
    \hyperlink{\link}{
        \begin{tikzpicture}
            \drawSquare{(0,0)}{1}{Blue}{\buttonWidth};
            \node at (0.5,0.5) {#2};
        \end{tikzpicture}
    }
}

\newcommand{\maxValue}[2]{
    \ifnum #1 < #2
        #2
    \else
        #1
    \fi
}

\newcommand{\minValue}[2]{
    \ifnum #1 < #2
        #1
    \else
        #2
    \fi
}

\newcommand{\boundValue}[3]{
    \minValue{\maxValue{#1}{#2}}{#3}
}

\newcommand{\loopValue}[3]{
    \ifnum #1 < #2
        #3
    \else
        \ifnum #1 > #3
            #2
        \else
            #1
        \fi
    \fi
}

\newcommand{\tryMove}[2]{
    \ifnum \ncalc{\x+#1} = 3
        \move{0}{#2}
    \else
        \move{#1}{#2}
    \fi
}

\newcommand{\move}[2]{
    \loopValue{\ncalc{\x+#1}}{0}{\ncalc{\gridX-1}};\loopValue{\ncalc{\y+#2}}{0}{\ncalc{\gridY-1}}
}

\newcommand{\controller}{
    % \def\x{#1}\def\y{#2}
    \tryMove{1}{0}
    % \move{0}{-1}
    \setlength{\tabcolsep}{-5.5pt} % Default value: 6pt
    \renewcommand{\arraystretch}{0} % Default value: 1
    \begin{tabular}{c c c}
        & \button{\move{0}{1}}{$\uparrow$} \\
        \button{\move{-1}{0}}{$\leftarrow$}
        & \button{\move{0}{-1}}{$\downarrow$}
        & \button{\move{1}{0}}{$\rightarrow$}
    \end{tabular}
}