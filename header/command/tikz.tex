\newcommand{\drawGrid}[3]{
    \foreach \n in {0,...,#1}
        \draw[line width = #3] (\n,0) -- (\n,#2);
    \foreach \n in {0,...,#2}
        \draw[line width = #3] (0,\n) -- (#1,\n);
}

\newcommand{\drawBorder}[3]{
    \ifnum \thetoricWorld = 1
        \draw[line width = #3, dashed] (0,0) -- (#1,0) -- (#1,#2) -- (0,#2) -- cycle;
    \else
        \draw[line width = #3] (0,0) -- (#1,0) -- (#1,#2) -- (0,#2) -- cycle;
    \fi
}

\newcommand{\drawSquare}[4]{ % 1 Offset - 2 Size - 3 Color - 4 Width
    \draw[line width = #4, color = #3, shift={#1}] (0,0) -- (0,#2) -- (#2,#2) -- (#2,0) -- cycle;
}

\newcommand{\level}{
    \begin{tikzpicture}[scale=\tikzScale]
        \drawLevelGrid;
        \drawTrees;
        \drawGoal;
        \drawDeathTraps;
        \drawCharacter;
    \end{tikzpicture}
}

\newcommand{\drawLevelGrid}{
    \ifnum \theshowGrid = 1
        \drawGrid{\gridX}{\gridY}{\gridWidth};%
    \fi
    \drawBorder{\gridX}{\gridY}{\borderWidth};
}

\newcommand{\drawCharacter}{
    % \drawSquare{(\x,\y)}{1}{\characterColor}{\characterWidth};
    \drawWalkingCharacter{(\x,\y)}{\characterColor}{\characterWidth};
}

\newcommand{\drawGoal}{
    \drawFlag{(\goal)}{\goalColor}{\goalWidth};
}

\newcommand{\drawFlag}[3]{ % 1 Offset - 2 Color - 3 Width
    % \node[shift={#1}] at (0,0) {\dash};
    \draw[line width = #3/1.2, color = RawSienna, shift={#1}, shift={(-0.05,-0.1)}] (0.5,0.6) -- (0.5,0.25) -- (0.4,0.25) -- (0.6,0.25);
    \draw[line width = #3, color = #2, shift={#1}, shift={(-0.05,-0.1)}] (0.5,0.95) -- (0.9,0.75) -- (0.5,0.55) -- cycle;
}

\newcommand{\drawSpikes}[3]{ % 1 Offset - 2 Color - 3 Width/Height
    \draw[line width = #3, color = #2, shift={#1}, shift = {(0.15,0.05)}] (0,0) -- (0.1,0.2) -- (0.2,0);
    \draw[line width = #3, color = #2, shift={#1}, shift = {(0.4,0.2)}] (0,0) -- (0.1,0.2) -- (0.2,0);
    \draw[line width = #3, color = #2, shift={#1}, shift = {(0.65,0.05)}] (0,0) -- (0.1,0.2) -- (0.2,0);
}

\newcommand{\drawTree}[3]{ % 1 Offset - 2 Color - 3 Width/Height
    % Trunk of the tree
    \draw[line width = #3, color = RawSienna, shift={#1}] (0.5,0.5) -- (0.5,0.1);
    
    % Foliage of the tree (Three ellipses representing the foliage)
    \draw[line width = #3, color = #2, shift={#1}] (0.5,0.7) circle (0.2);
    \draw[line width = #3, color = #2, shift={#1}] (0.7,0.5) circle (0.2);
    \draw[line width = #3, color = #2, shift={#1}] (0.3,0.5) circle (0.2);
}

\newcommand{\drawWalkingCharacter}[3]{ % 1 Offset - 2 Color - 3 Size
    \isEven{\ncalc{\x-\y}}
    \ifnum \theres = 1
        \def\behindArm{(0.7, 0.35)}
        \def\beforeArm{(0.3, 0.35)}
        \def\behindLeg{(0.4, 0.1)}
        \def\beforeLeg{(0.6, 0.1)}
    \else
        \def\behindArm{(0.3, 0.35)}
        \def\beforeArm{(0.7, 0.35)}
        \def\behindLeg{(0.6, 0.1)}
        \def\beforeLeg{(0.4, 0.1)}
    \fi
    % Behind Arm
    \draw[line width = #3, color = #2!20!black, shift={#1}] (0.5,0.5) -- \behindArm;
    
    % Behind Leg
    \draw[line width = #3, color = #2!20!black, shift={#1}] (0.48,0.2) -- \behindLeg;

    % Head
    \draw[line width = #3, color = #2, shift={#1}] (0.5,0.7) circle (0.1);
    
    % Body
    \draw[line width = #3, color = #2, shift={#1}] (0.5,0.35) ellipse (0.07 and 0.15);
    
    % Before Arm
    \draw[line width = #3, color = #2, shift={#1}] (0.5,0.5) -- \beforeArm;
    
    % Before Leg
    \draw[line width = #3, color = #2, shift={#1}] (0.52,0.2) -- \beforeLeg;
}

\newcommand{\drawTrees}{
    \foreach \tree in \trees{
        \drawTree{(\tree)}{\treeColor}{\treeWidth};
    }
}

\newcommand{\drawDeathTraps}{
    \foreach \spike in \spikes{
        \drawSpikes{(\spike)}{\spikeColor}{\spikeWidth};
    }
}

\newcommand{\button}[2]{%
    \def\link{#1}%
    \hyperlink{\link}{
        \begin{tikzpicture}
            \drawSquare{(0,0)}{1}{Blue}{\buttonWidth};
            \node at (0.5,0.5) {#2};
        \end{tikzpicture}
    }
}

\newcommand{\levelButton}[1]{
    \def\link{run:./level_#1.pdf}%
    % \href{\link}{Level #1}
    \href{\link}{
        \begin{tikzpicture}
            \drawSquare{(0,0)}{1}{Blue}{\buttonWidth};
            \node at (0.5,0.7) {Level};
            \node at (0.5,0.3) {#1};
        \end{tikzpicture}
    }
}

% \newcommand{\fileButton}[2]{
%     \def\link{#1}%
%     \href{run:./#1}{
%         \begin{tikzpicture}
%             \drawSquare{(0,0)}{1}{Blue}{\buttonWidth};
%             \node at (0.5,0.5) {#2};
%         \end{tikzpicture}
%     }
% }